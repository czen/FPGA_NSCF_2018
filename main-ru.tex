\documentclass[utf8]{psta}% 
% Пожалуйста, уточните классификация Вашей статьи согласно УДК, ББК и MSC
\subjclass[UDC]{004.4'422}
\subjclass[BBC]{32.972.11}
\subjclass[2010]{68N20}

% Заглавие работы
\title[Статическая классификация программ для ПЛИС]{Метод статической классификации участков программы для отображения на ПЛИС}
%\title[]{} % или с коротким вариантом для колонтитула

% Фамилия, запятая, имя отчество автора
\author{Баглий, Антон Павлович}
% Организация, в которой выполнена статья или её часть автором
\address{Южный федеральный университет}
% Электронный адрес автора
\email{abagly@sfedu.ru}
% Какими грантами поддержана работа автора
\thanks{}
%  краткая информация, в свободной форме  презентующая официальный статус автора, его научные интересы и достижения. 
\info{}
% фотография, позволяющая узнать автора в толпе участников любой конференции
%\image{}
\image{nobody}
\orcid{}    
% Аналогично для каждого из остальных авторов

% Пару строчек ключевых слов и фраз для поиска
\keywords{}
\begin{abstract}
   Приводятся результаты продолжающегося исследования по поиску способ классификации участков высокоуровневой программы для отображения на ПЛИС. Описываются данн  На основе разработанного метода статического профилирования программы в высокоуровневом представлении строится алгоритм эмуляции абстрактной перенастраиваемой архитектуры с целью более точной оценки скорости работы схемы, построенной по выбранному участку программы.  
\end{abstract}

% Все метаданные должны также присутствовать на английском языке, 
% заключённые в  \selectlanguage{english}...,\selectlanguage{russian}: 

\selectlanguage{english} 
% All the same in English 
\title[Static program classification for FPGA]{Technique for static classification of program fragments to be mapped onto FPGA}
% Last name, coma other names 
\author{Bagly, Anton Pavlovich}
% Organisation, where the work done
\address{Southern federal university}
% author email
\email{abagly@sfedu.ru}
% support notes
\thanks{}
% Other information about author only on paper language 
%\info{} %
% author photo
%\image{}
%\orcid{}
% Repeat the same fore each of other authors
%
\begin{abstract}
РАЗДЕЛ В РАЗРАБОТКЕ
\end{abstract}
\selectlanguage{russian} % Не забывайте отметить возврат на русский язык
% Для локального переключения на другой язык используйте команду 
% \foreignlanguage{english}{Text in English}

\begin{document}           
\maketitle   
%%%%%% Текст статьи может использовать 
\section*{Введение}

\section{Постановка задачи} 

классификация участков кода для отображения на ПЛИС:

\begin{itemize}
    \item Поиск гнезд циклов, которые работают с небольшим объемом данных
    \item Поиск участков программы с характеристиками пригодными для ПЛИС
    \item как искать:
    \begin{itemize}
        \item поверхностный анализ циклов - очень узки класс программ
        \item анализ потока данных?
        \item метрики (простой стат. анализ)
        \item симуляция поведения программы, эмуляция выполнения
        \item профилирование во время выполнения
        \item низкоуровневая эмуляция процессора на программе, сбор данных о работе с памятью и т.п.
    \end{itemize}
   \item Поиск участков с повторяющимися вычислениями
   \item выражения?
   \itemоператоры?
\end{itemize}

\section{Статический анализ для оценки пригодности для ПЛИС}

\section{Быстрая аппроксимация времени выполнения и возможности выполнения на ПЛИС} 

\cite{Sherwood2002}

за счет:

\begin{itemize}
   \item  эмулятора на базе интерпретатора С, который бы подсчитывал время с учетом доступных вычислительных модулей. Интерпретатор должен работать за счет обхода управляющего графа по простым блокам и их выполнения в порядке, допускаемом инф. зависимостями. При этом для выполнения определенных операторов задействуются виртуальные вычислительные модули, количество и тип которых задается заранее.
   \item возможности указанного интерпретатора пройти вперед до определенного блока кода (для ускорения). Можно пытаться задействовать LLVM и считать простые блоки в нем, лишь бы получилось сделать fast-forward до нужного блока
   \item использования simpoints для хорошей оценки времени работы на эмулируемой архитектуре
   \item вероятность того, что проект не поместится на ПЛИС, можно оценить по результатам работы интерпретатора (сколько виртуальных вычислителей было задействовано, сколько можно еще задействовать)
\end{itemize}

\section{эмуляция абстрактной архитектуры}

\cite{Takamaeda-Yamazaki2014}
  
   

%Проверяйте, пожалуйста (URL) в списке литературы \cite{PSTAmanual}!% ссылка на источник литературы
\section{}
\subsection{}
%\citеs{,,} % ссылка на несколько источников

%\begin{figure} % Never fix the place!
%\includegraphics{pic} % Без расширения, должно быть jpg, png или pdf
%%\includegraphics[width=10cm]{pic} % Если по ширине страницы
%\caption{}
%\label{}
%\end{figure}

%%%%% Нумерация библиографии в порядке цитирования, инициалы перд фамилией. 
% 
\bibliographystyle{amsalpha}
\bibliography{bibliography.bib}

\end{document}

